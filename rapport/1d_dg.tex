\documentclass[a4paper,10pt,draft]{article}

\usepackage{ucs}
\usepackage[utf8]{inputenc}
\usepackage{amsmath}
\usepackage{amsfonts}
\usepackage{amssymb}
\usepackage{amsthm}
\usepackage{caption}
\usepackage[english]{babel}
\usepackage[T1]{fontenc}
\usepackage[pdftex]{graphicx}
\usepackage{ stmaryrd }

\usepackage{hyperref}

\title{1D DG method}
\author{Etienne PEILLON}
\date{15/05/2019}

\begin{document}
 \maketitle
 
 \section{Problem}
 
 We consider the following problem:
 
 \begin{equation} \label{eq:initial}
  \alpha u - \mu \Delta u = f
 \end{equation}

 on the interval $\Omega = ]a,b[$ with one of the three following boundary conditions:
 \begin{itemize}
  \item Dirichlet conditions: $u(a) = u_a^D$ and $u(b) = u_b^D$;
  \item Neumann conditions: $u'(a) = u_a^N$ and $u'(b) = u_b^N$;
  \item Mixed conditions: $u(a) + \gamma_a u'(a) = u_a^M$ and $u(b) + \gamma_b u'(b) = u_b^M$.
 \end{itemize}

 \section{Discontinuous discretization}
 
 To discretize the problem we will follow some step:
 \begin{enumerate}
  \item weak formulation of the problem compatible with the method
  \item writing of the basic matrix
  \item implementation
 \end{enumerate}
 
 The goal of this report is simple: give a little overview of the DG method with understandable but 
extendable notations.

 \subsection{Weak formulation}
 
 Let be a subdivision $a = x_0 < x_1 < \dots < x_{K-1} < x_K = b$ of I. We write $C_n = 
]x_{n-1},x_n[$, so that we have $K$ cells, and $\mathcal{E}_h = \{ C_n \}$ where $h = \max 
\limits_{n} |C_n|$.
 
 We consider the broken Sobolev space, highly depending on the subdivision of the domain (see 
Riviere):
 \begin{equation}
  H^1(\mathcal{E}_h) := \{ v \in L^2(\Omega) \quad that \quad \forall E \in \mathcal{E}_h, v|_E \in 
H^1(E) \}
 \end{equation}

 Denote that $H^1(\Omega) \subset H^1(\mathcal{E}_h)$.
 
 \paragraph{}
 The strong idea of discontinuous Galerkin method is to approximate the problem, not with an 
approximation of $H^1(\Omega)$ as for continuous Galerkin method, but with an approximation of 
$H^1(\mathcal{E}_h)$.


\paragraph{}
Let be $v \in H^1(\mathcal{E}_h)$, and multiply \ref{eq:initial} by it and then use Green formula 
(also known as integration by parts in 1D). We take care that Green formula works only on each cell:

\begin{equation}
 \sum \limits_{n=1}^K \int_{C_n} (\mu u'v' + \alpha uv) d\omega - \mu [u'(x)v(x)]_{x_{n-1}}^{x_n} = 
\sum \limits_{n=1}^K \int_{C_n}  f\ v\ d\omega
\end{equation}

Here exists different methods in the literature. In Riviere, terms are added to create a bi-linear 
symmetric coercive and continuous form on left side and a linear conituous form on the right side, 
so that you can prove existence and uniqueness with functional analysis theorems. The approach is 
very interesting in the theorical way, but complicated for just a simple implementation.

On the other hand, you can use the following approach, which is less general, but more 
understandable as first view of the subject:

\paragraph{}
Because $v \in H^1(\mathcal{E}_h)$, we assume $u \in H^1(\mathcal{E}_h)$ for the resolution of the 
problem, but we want equality of the solution on the commun edges of each cell, wich means we have 
a bad-defined value at $u'(x_n)$ in our case, wich is the \emph{numerical flux}.

A solution consists on assuming that numerical flux equals one unique value:

$$
\widehat{\nabla u(x)} = \frac{\beta}{2} \llbracket u \rrbracket + \overline{\nabla u(x)}
$$

where the definition of the different term are given in Armand's paper p.15.

\subsection{$H^1(\mathcal{E}_h)$ discretization}

 
 
 
\end{document}
